\documentclass{standalone}

% Preamble
\begin{document}

\subsection{What happens when the dimension of $A$ is infinite ?}
When the square polynomial system is not zero-dimensional, then dimension of $A$ is infinite and this is clearly indicated by the Groebner calculation. The Bezout process, no matter the dimension of $A$ is finite or infinite, always provide companion matrices of finite size.
What we observe in our experiments is that whenever the dimension of $A$ is infinite - as indicated by the Groebner computation - then two of the three items of the test \ref{testing_the_result} are still valid, that is to say, we still have $f(X) = 0$ (item 2) and the eigenvalues of the companion matrices are still roots of $f$ (item 3).




\end{document}
