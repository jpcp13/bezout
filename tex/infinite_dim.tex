\documentclass{standalone}

% Preamble
\begin{document}

\subsection{What happens when the polynomial system is not zero-dimensional ?}
It may happen that the square polynomial system is not zero-dimensional ; the dimension of $A$ is then infinite and this is indicated by the Groebner calculation. The Bezout process, no matter the dimension of $A$ is finite or infinite, always produces matrices $B^{(j)}$ of finite size ; in the case when the dimension is infinite, we still call the matrices $X_j = B^{(j)}{B^{(0)}}^{-1}$ companion matrices.
What we observe in our experiments is that, whenever the dimension of $A$ is infinite,  we still have $f(X) = 0$ (item 2 of test \ref{testing_the_result}) and the eigenvalues of the companion matrices are still roots of $f$ (item 3 of test \ref{testing_the_result}).

\end{document}
